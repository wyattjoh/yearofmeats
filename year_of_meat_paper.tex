\documentclass{article}
\usepackage{mla13}

% \doublespacing

\firstname{Wyatt}
\lastname{Johnson}
\professor{Lisa Haynes}
\class{English 124 B1: Literary Analysis}
\title{Essay Assignment: Outline}

\sources{year_of_meat_paper}

%% ASSIGNMENT DESCRIPTION

% Outline: We will discuss the format of the outline in class. Your outline
% should include the following information, in full, grammatical sentences:

\begin{document}
\makeheader
% 3. An initial thesis for your paper (this, too, may evolve as you write—which
% is good!)
\textbf{Thesis:} The practices and goals of the Beef Export and Trade
Syndicate and beef industry as described in Ruth Ozeki's book
\citetitle{ozeki1998my} and their use of antibiotics and hormones directly
influences the drastic changes to the characters physically, sets the tone,
and builds up relationships in the book in both their physical and emotional
states.

% 1. What theme will you analyze in the text? ? (You may, and likely will, narrow
% your focus as you write)
The main characters in this book that undergo the most change are Jane
Takagi-Little, and Akiko Ueno. Through the analysis of these characters, I
hope to glean the physical, physiological, and emotional changes that occur
due to their interaction with the ideals and drugs used by the industry.

% 2. Why did you choose this theme: a) why is it important to discuss it, and b)
% why is it of interest to you, personally? (Did you find something
% interesting, confusing, pleasurable, problematic, personally resonant etc.?)
I felt that this theme of character development as a result of the influence
from the beef industry was interesting to me as I feel that the way that the
changes occurred in some of the characters did reflect in me as I was
reading the book, and I feel that due to this connection, I can better
relate. The importance of the changes highlight the struggle that takes
place in-between what we think is safe or not safe to eat. It was
interesting to discover how Jane developed and learned so much about the
beef industry, specifically relating to her own personal health ``it
didn't’t take me long to stumble across DES. It was a discovery that
ultimately changed my relationship with meats and television. It also
changed the course of my life'' \cite{ozeki1998my}. When describing the
resulting damage of the DES to her uterus, she referred to the ideal one
looking like ``head of a bull, with the fallopian tubes spreading and
curling like noble horns'', while as a result of the hormones before birth
``[t]he left side of the bull’s broad forehead was caved in, less
triangular, as though my uterus had been coldcocked'' \cite{ozeki1998my}. It
is also a interesting the effects that Akiko feels as a result of the
consumption of the meat, where it would begin ``in her stomach, like an
animal alive, and would climb its way back up her gullet, until it burst
from the back of her throat'' \cite{ozeki1998my}. When she had instead eaten
the Australian lamb, she ``went to the bathroom and waited, but the animal
inside her was quiet'' \cite{ozeki1998my}. 

% 4. What literary elements of the texts will you analyze to show how they
% develop the theme?
I will be focusing on the development of the characters through their
physical descriptions, the way that they interact with their authority
figures, and the actions that they are taking.


% 5. Evidence you can draw on to support your argument

\makeworkscited
\end{document}
