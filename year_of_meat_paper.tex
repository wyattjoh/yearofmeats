\documentclass{article}
\usepackage{hyperref}
\usepackage{mla13}

\firstname{Wyatt}
\lastname{Johnson}
\professor{Lisa Haynes}
\class{English 124 B1}
\title{Cooperate Greed in Ruth Ozeki's \citetitle{ozeki1998my}}

\sources{year_of_meat_paper}

\begin{document}
\makeheader

% TODO: Talk about storytelling, tone, and setting somewhere... Ozeki uses story, tone, and setting to show how the BEEF-EX represents cooperate greed.
One of the most influential entities in Ruth Ozeki's book
\citetitle{ozeki1998my} is the Beef Export and Trade Syndicate (or BEEF-EX),
a large lobby group representing the interests of ``beef, pork, lamb, goat,
horse—as well as livestock producers, packers, purveyors, exporters, grain
promoters, pharmaceutical companies, and agribusiness groups'' \cite[Ch.
1]{ozeki1998my}. As we learn about the operations of these industries
through characters at different levels of involvement with the meat
industry, we find many instances of corporate greed. The industry is willing
to sacrifice the well being of their customers during the production of the
BEEF-EX sponsored show, ``My American Wife!'' We also learn about the major
role that pharmaceutical companies play in the production of meat, primarily
in the use of Diethylstilbestrol (or DES). We learn that these sometimes untested drugs
were used to simply improve efficiency, but unfortunately had the side
effect of causing massive damage to a female consumer's reproductive system.
\citeauthor{ozeki1998my} uses storytelling, tone, and setting to show how
maximizing profit comes at the cost of morals.

% Focus throughout on how literary techniques develop a theme, rather
% than factual evidence

% TODO: Refocus first sentence on a literary thingy
% When it comes to a company's quarterly statement, making profit is more important than the health of the masses. Until 1959, the use
% of DES was quite rampant in the poultry industry as it enhanced many traits
% that made the meat more desirable. At that time, the issues with
% DES became clear when ``dogs and males from low-income families in the South
% were developing signs of feminization after eating cheap chicken parts and
% wastes from processing plants'' \cite[Ch. 6]{ozeki1998my}, and the Federal Drug
% Administration (FDA) banned it for use with poultry. In 1954, the FDA approved the use of DES for cattle,
% heralding it as ````miracle'' and ``a revolution in the cattle industry,''''
% \cite[Ch. 6]{ozeki1998my}. As a result, DES was then ``used by more than 95
% percent of U.S. cattle feeders to speed up production'' \cite[Ch. 6]{ozeki1998my}.
% When the evidence became clear that there were many medical reasons to
% stop using DES, it still took the FDA until 1979 to actually ban its use in
% livestock production.

%   Storytelling
%
% - characters viewpoint on BEEF-EX
% - Are the characters described such that the reader identifies with them?
% - Jane and her mother's relationship

Storytelling is used in \citetitle{ozeki1998my} to provide personal insight
into the workings of BEEF-EX while also developing a character that the
reader can identify with. The main character in the story, Jane, is a
coordinator for the new Japanese television show ``My American Wife'' where
she must travel around the United States to find families that can be
featured on the show. The role of the show is to ``foster among Japanese
housewives a proper understanding of the wholesomeness of U.S. meats.''
\cite[Ch. 1]{ozeki1998my} which would further BEEF-EX's associated
companies. Initially, Jane is innocent to the idea that the beef that she is
marketing is hiding a dark secret. When it is later revealed to her, she
only eats organic red meats if she does at all \cite[Ch. 9]{ozeki1998my}.
The believable nature of Jane makes it really easy for a reader to
personally identify with the issues that she encounters. Through the story,
we are exposed to a variety of situations that are used by the author to
provide factual evidence to support the corruption both in the meat and in
the processes of BEEF-EX. One of the most powerful and longest periods that
we learn about information through story was that of Jane's own attempts at
having children, and her discovery of DES which ``ultimately changed my
relationship with meats and television. It also changed the course of my
life'' \cite[Ch. 6]{ozeki1998my}. When describing the resulting damage of
Jane's uterus from the DES that her mother took during her pregnancy, she
referred to the ideal one looking like ``head of a bull, with the fallopian
tubes spreading and curling like noble horns'', while as a result of the
hormones before birth ``[t]he left side of the bull’s broad forehead was
caved in, less triangular, as though my uterus had been coldcocked''
\cite[Ch. 7]{ozeki1998my}. The references to the ``head of a bull''
\cite[Ch. 7]{ozeki1998my} might also be a symbol of strength, or power. As a
result of the DES, or the cooperation, Jane is left with a weakened sense of
strength. The changes then described to her mother will highlight the
difference over generations, that her mother was not even aware of these
issues back then. In fact, her mother didn't even know the drugs that she
was given, she ``try everything. Some vitamin, some Doctor Ing-san
medicine'' \cite[Ch. 11]{ozeki1998my}. This shows her mother's blind faith
in the doctor and in the system, eventually resulting in being given DES.
The doctors prescribed this even when studies that were released at the same
time when it was approved for cattle, DES ``showed a significant increase
not only in miscarriages but also in premature births and infant deaths''
\cite[Ch. 6]{ozeki1998my}. It was not common knowledge as the very same drug
that was used by pregnant women to strengthen their children was ``even used
as a morning-after pill to terminate pregnancy'' \cite[Ch. 6]{ozeki1998my}.
This can be contrasted to the attitude that Jane had when she was prescribed
Tace for her pains after the miscarriage. Armed with the knowledge of the
drug industry, Jane decided to look it up, just to find that it could
actually be quite harmful to someone with her condition! This relates quite
nicely to the idea of the trust that the meat industry has, or rather had
relating to the safety of their products. It was the simple fact that the
corporations ignored the evidence, as they were making profit. Jane's
development of the relationships with the people that she filmed provided
another insight to the industries goals. Lara and Dyann in the novel are
used as a pipeline to describe factual and historical evidence of the food
and drug industry and the lengths that some people go to avoid the issues
that are caused by them. Dyann went so far as to state that they could
``never eat it[meat] the way it’s produced here in America. It’s unhealthy.
Not to mention corrupt, inhumane, and out of control'' \cite[Ch.
8]{ozeki1998my}. Another character that provided a lot of information on the
industry was Bunny and Rosie. Their role was more of a direct effect of the
results of the industry as they lived on a cattle ranch. As a result of the
exposure to DES, Bunny's daughter, Rosie, suffered greatly. Rosie underwent
puberty at age 5, developing traits that would have taken many more years to
develop which was a direct result of the extremely high amounts of DES in
the environment. Jane later used footage of Rosie to spark a controversy
relating to the illegal use of DES in 1991, which was still being used 12
years after it was supposedly banned in livestock production. This of course
as motivated by profit, as without the use of these artificial growth
accelerants, they could have gone out of business like Jane's grandparents
which had lost ``the family dairy farm to hormonally enhanced cows''
\cite[Ch. 6]{ozeki1998my}. Through the story telling by the characters in
\citetitle{ozeki1998my}, we are focused by \citeauthor{ozeki1998my} into
believing that BEEF-EX utilizes its power over the American people which is
then abused for profit.

% tone
%
% - Changes from looking foward to the futre, optimism, innocence to a feeling
%   of lack of control and dispair

The tone of the novel changes throughout the book and follows opinion that
Jane has on the industry. In the beginning, Jane seems to look forward to
``My American Wife!'' which would ``land [her] a job and keep [her] both
meat-fed and employed for over a year'' \cite[Ch. 1]{ozeki1998my}. Her naive
outlook on the meat industry can be seen as a symbol for the innocence of
the American public with relation to the industry. When we learn more about
the use of hormones and drugs used in the book, the tone becomes much darker
and threatening. The tone is established first with Jane's new job and
relationship. Before this new job, Jane was freezing, and in need of a job.
At this point as well, we meet Sloan, which later causes Jane to become
pregnant. This pregnancy is interpreted as a very positive thing by Jane, as
she had encountered trouble previously with attempting to have a child. We
also learn about Akiko, who we first see as a very devoted wife that will do
anything necessary to please her husband. This relationship devolves however
at the cost of being beaten by her husband. This then results in her fleeing
to America to have her child. Jane's experience is then topped off by having
a miscarriage as a direct result of the industries drug practices. The Tokyo
office describes the ``Ideal American Family'' as being free of ```1.
Physical imperfections 2. Obesity 3. Squalor 4. Second class peoples''
\cite[Ch. 1]{ozeki1998my}. This makes BEEF-EX seem very mechanical. The
Tokyo office does not bother to care or worry about how the staff might
interpret their requests. Their sole and only purpose is to convince the
Japanese public that they should buy American beef because they are using
the cast to make it seem like as a result they will have ``Perfect
families'' \cite[Ch. 1]{ozeki1998my}. This portrayal that they must strive
towards. It is not their intention at all to portray any form of an
``American Wife'' as so ironically is their title of the show. It was also
quite ironic how the most important point they stressed in all caps was that
the ``MOST IMPORTANT THING IS VALUES, WHICH MUST BE ALL-AMERICAN'' \cite[Ch.
1]{ozeki1998my}. This of course clashing with their undesirable and
desirable points, as America is certainly a country full of ``Physical
imperfections'' and of ``Second class people'' \cite[Ch. 1]{ozeki1998my}.

%% SETTING

% What concrete details about the setting does the narrator provide?
% • What imagery is used: what details appear to one or more of the five senses?
% • What is the setting's orientation to the contemporary “real world”? Does the text invert of
% subvert some aspect of the real world? (e.g. Magic Realism: combines the objectively real
% with the surprise of the unreal or unexpected; Science Fiction: may include technologies etc.
% that do not exist; Metafiction: focuses on the text itself as the testing ground of the real).
% • Does the author use a conventional “type” of setting? (e.g. Utopia, dystopia, fantasy world,
% mythical place, possible future, historical past, etc.)
% • Do the physical and/or nonphysical settings affect characters' actions and/or the text's plot?
% • Does the setting and/or the way the setting is described help to establish mood? (e.g. “a
% bright, sunny day” vs. “a dark and stormy night”; “a bright, sunny day” vs. “blazing midday
% sun”)
% • Is the setting associated with specific concepts? (e.g. place of worship: religion; school:
% education; store: commerce, exchange; castle: class, power, history, etc.)

% Talk about... setting... yeah, thats it.

%% CONCLUSION

Essentially, I am arguing that the corporate entities as described within
Ruth Ozeki's book \citetitle{ozeki1998my} act with a profit first approach
which results in physical and psychological damage to the consumers of its
products and content. The drugs were used, even though clear evidence that
they were unfit for human consumption. These drugs were then later turned
around and fed to the consumer directly, with the thought that some
marketing jazz would be a sufficient reason to use it. Everything from the
drugs that they marketed, to the food they produced, all tainted by
corporate greed. With food being such an important component of everyone's
lives, I think that greater care should be taken to ensure that entities
that are controlling them act with a consumer first approach instead.

\makeworkscited
\end{document}
