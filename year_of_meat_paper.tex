\documentclass{article}
\usepackage{mla13}

% \doublespacing

\firstname{Wyatt}
\lastname{Johnson}
\professor{Lisa Haynes}
\class{English 124 B1: Literary Analysis}
\title{Corporations, affecting more than your bottom line}

\sources{year_of_meat_paper}

%% ASSIGNMENT DESCRIPTION

% Outline: We will discuss the format of the outline in class. Your outline
% should include the following information, in full, grammatical sentences:

\begin{document}
\makeheader
% 3. An initial thesis for your paper (this, too, may evolve as you write—which
% is good!)
% \textbf{Thesis:} The practices and goals of the Beef Export and Trade
% Syndicate and beef industry as described in Ruth Ozeki's book
% \citetitle{ozeki1998my} and their use of antibiotics and hormones directly
% influences the drastic changes to the characters physically, which is
% illustrated by the tone, and the relationships.

One of the most influential entities in Ruth Ozeki's book
\citetitle{ozeki1998my} is the Beef Export and Trade Syndicate (or BEEF-EX),
a large lobby group representing the interests of ``beef, pork, lamb, goat,
horse—as well as livestock producers, packers, purveyors, exporters, grain
promoters, pharmaceutical companies, and agribusiness groups''
\cite{ozeki1998my}. As we get a very wide angle view of many parts of this
collection of industries, we are given a glimpse into the theme of corporate
greed, and how the corporations will go to protect their ideals and profit
margins. BEEF-EX in the novel is a fictional entity, but it's existence is
based in reality, relating to the food industry at the time. In order for
the lobby group to convince the wives of Japan to consume more meat, they
choose to portray an ideal American family. For BEEF-EX to produce enough
meat to sell, the use of hormones and other drugs became the standard
practice. These drugs later were revealed to cause major side effects, but
were continued to be used in spite of the data.

% Topics:
%   1. Ideal American Family
%   2. DES, and it's affects

The concept of the ideal American family is reinforced at multiple occasions
by BEEF-EX can be seen as quite unrealistic. They reinforce their decision
sole on the market research, eliminating characteristics such as ``1.
Physical imperfections 2. Obesity 3. Squalor 4. Second class peoples''
\cite{ozeki1998my}.

% 1. Talk about memos
% 2. Talk about John, issues with families chosen

% 1. What theme will you analyze in the text? ? (You may, and likely will, narrow
% your focus as you write)
% 2. Why did you choose this theme: a) why is it important to discuss it, and b)
% why is it of interest to you, personally? (Did you find something
% interesting, confusing, pleasurable, problematic, personally resonant etc.?)
% 4. What literary elements of the texts will you analyze to show how they
% develop the theme?
% 5. Evidence you can draw on to support your argument

% DES and their effects

% I felt that this theme of character development as a result of the influence
% from the beef industry was interesting to me as I feel that the way that the
% changes occurred in some of the characters did reflect in me as I was
% reading the book, and I feel that due to this connection, I can better
% relate.
As a reader, I felt very moved by the revelations in this book relating to
the food industry.
The importance of the changes highlight the struggle that takes
place in-between what we think is safe or not safe to eat. It was
interesting to discover how Jane developed and learned so much about the
beef industry, specifically relating to her own personal health ``it
didn't take me long to stumble across DES. It was a discovery that
ultimately changed my relationship with meats and television. It also
changed the course of my life'' \cite{ozeki1998my}. When describing the
resulting damage of the DES to her uterus, she referred to the ideal one
looking like ``head of a bull, with the fallopian tubes spreading and
curling like noble horns'', while as a result of the hormones before birth
``[t]he left side of the bull’s broad forehead was caved in, less
triangular, as though my uterus had been coldcocked'' \cite{ozeki1998my}.
These changes then described to her mother will highlight the difference
over generations, that her mother was not even aware of these issues back
then. In fact, her mother didn't even know the drugs that she was given,
showing her blind faith that the doctor would give her something safe. This
contrasts the attitude that Jane did when she was prescribed something for
her pains after the miscarriage, where she questioned it, looked it up, just
to find that it could actually be quite harmful to her! This will relate
quite nicely to the idea of the trust that the meat industry has or had
relating to the safety of their products.

The mood of the novel also changes from the start from a lighter seeming
novel to being quite dark. In the beginning, Jane seems to be looking
forward to the employment, which would keep her fed and employed for over a
year. When we learn more about the use of hormones in the book, we become
much darker, with Jane's miscarriage, and Akiko's husband beating her. This
sets us up for the big reveal of Jane's documentary focusing on the meat
industries practices at the end of the novel.

Jane's relationships really develop out the other viewpoints and the facts
relating to the industry. Even her relationship with Sloan develops quite
interestingly. The wording that is used in their communication near the end
and at the beginning, especially after the miscarriage which gets Sloan
obviously quite upset that she had been in the feeding lot. These events of
course triggered by the incorrect development of Jane's reproductive organs
due to the drugs as designed by BEEF-EX. The relationships with some of the
characters in her shows, such as Lara and Dyann, are used as methods to
describe factual and historical evidence of the food and drug industry and
the lengths that some people go to avoid the issues that are caused by them.
The role that Bunny and Rosie played in providing direct evidence of the
effects of the drugs used in feed lots, which was later used to great effect
in the documentary developed by Jane.

I found it quite entertaining the attitude that Jane took starting near the
middle of the book of essentially ignoring the requests of John and BEEF-EX
regarding the ``Pork is Possible, but Beef is Best!'' \cite{ozeki1998my}
motto. While this doesn't specifically highlight the hormones and
antibiotics used, these actions are almost taken in direct response to her
finding out more about the industry and wanting to reveal more of the
``real'' world of beef. This will then be supported by the descriptions of
her actions taken in the feed lots and the killing/packing plant.

It was also very surprising to me the effects of these hormones and drugs
had on developing children. This was a very important focus in this book, on
fertility. Both Jane and Akiko both had issues with conception, where Jane's
attempts seemed to have failed completely. This development will be focused
on Jane's past, the feed lots that she visited, the miscarriage, and the
issues that Akiko was facing.

Pulling everything together, the characters, their relationships, and the
tone really set the stage for arguing the negative affects and resulting
changes on the characters as a result of the drugs and practices used by
BEEF-EX.
\makeworkscited
\end{document}
